\chapter{Introduction} \label{chap:introduction}

The digitization of health services and clinical processes has resulted in the
healthcare industry generating an ever-increasing amount of textual data,
encompassing electronic health records, clinical notes, medical reports, and
discharge letters among many others. While structured data is frequently used
for health economics and registries, the aforementioned unstructured clinical
narratives are preferred by physicians to record patients' clinical information
due to their flexibility and efficiency, and make up to 40\% of the data
generated in current hospital systems~\cite{wang2018clinical,
dalianis2009stockholm}. The substantial potential of narrative text data to
support clinical applications was recognized early~\cite{sager1994natural,
borst1991textinfo,friedman1995architectural} and more recently, research efforts
have been directed towards developing medical applications assisted by
artificial intelligence (AI). Prominent applications include decision support
systems that assist healthcare professionals in their tasks, alleviating their
workload and providing better treatments for patients~\cite{zhou2022natural}.

However, the unstructured nature of textual data and the intricacies of the
biomedical field pose significant challenges for leveraging its potential. In
such a context, natural language processing (NLP) methods could structure that
information to support downstream clinical applications. Recent advancements in
NLP brought about by large-scale pre-trained language models based on the
Transformer~\cite{vaswani2017attention} architecture, present new possibilities
for extracting and analyzing the knowledge contained within the clinical texts.
Through extensive self-supervised training on vast corpora of text, a model can
acquire valuable representations of a language, producing highly effective
language models. 

The success of Transformer-based models like BERT (Bidirectional Encoder
Representations from Transformers)~\cite{devlin2019bert} and its improved
version RoBERTa~\cite{liu2019roberta}, can be largely attributed to the use of
transfer learning expressed in the pretrain-finetune paradigm. In this paradigm,
a model initially goes through a resource-demanding training process, i.e.
\textit{pre-training}, using general-purpose textual data to learn the language
structure. This pre-training phase is self-supervised, eliminating the need for
labeled data by utilizing objectives like masked language
modeling~\cite{devlin2019bert}. The model is then \textit{fine-tuned} for
various tasks through a second, more cost-effective training round using a
smaller, labeled, and task-specific dataset that adjusts the model's weights to
fit the specific task and application domain at hand.  

Direct application of general-purpose language models to a specific domain
result in suboptimal performance due to significant distributional differences
between general and target domains. Even within the same language,
domain-specific language can vary significantly from everyday language, leading
to the need of domain-specific models~\cite{arefeva2022tourbert}. This
particularly holds for the medical domain, where the language is highly
specialized and complex. Medical language features numerous acronyms that are
crucial for saving time and space, yet they can be ambiguous and require context
to be understood. Spelling errors are common, and there is an abundance of
abbreviations~\cite{tayefi2021challenges}. Moreover, the medical vocabulary is
highly specialized, as it is not typically used in everyday language, making it
unfamiliar to those outside the medical profession. When the target domain, such
as medicine, differs considerably from the pre-training data, models can be
improved by an additional phase of domain-adaptive training using large,
domain-specific corpora with the same pre-training objectives. 

Such specifically designed medical language models hold significant promise for
enhancing the efficiency and precision of medical document
handling~\cite{beltagy2019scibert, huang2019clinicalbert, peng2019transfer,
lee2020biobert}. For the German medical domain, the effectiveness of such models
has been demonstrated by Bio-GottBERT~\cite{lentzen2022critical} and
medBERT.de~\cite{bressem2024medbert}. However, the availability of open-source
biomedical corpora large enough for domain adaptation is limited, primarily due
to the sensitive nature of health-related data, and is largely confined to the
English language, given its established status as the language of science.
Despite these obstacles, advancing medical language models remains crucial, as
they have the potential to manage the large volumes of text produced in
hospitals every day.

\section{Related Work}

Past developments in medical NLP research have seen the creation of mature
systems for extracting information from English clinical texts like
MetaMap~\cite{aronson2010overview}, cTAKES~\cite{savova2010mayo},
MedLEE~\cite{friedman1995architectural, friedman2000broad} and
CLAMP~\cite{soysal2018clamp}, which have been widely deployed in clinical
settings. These systems have been used for various tasks such as named entity
recognition (NER), relation extraction, and information retrieval. Additionally,
open competitions such as Informatics for Integrating Biology and the Bedside
(i2b2)~\cite{uzuner20112010}, National NLP Clinical Challenges
(n2c2)~\cite{henry20202018, stubbs2019cohort}, and CLEF eHealth
\cite{crestani2019experimental} challenge from the Conference and Labs of the
Evaluation Forum (CLEF) promote data and model sharing, further advancing the
medical NLP field. The systems developed to date encompass rule-based,
machine-learning-based, and hybrid models. While rule-based methods were
essential in early developments, the performance of these systems is limited by
their reliance on hand-crafted rules and lexicons, which are difficult to
maintain and generalize across different clinical settings. 

In order to overcome these challenges, current research emphasizes
machine-learning techniques. In particular, deep-learning approaches like
recurrent neural networks and convolutional neural networks have been widely
used in recent years due to their ability to achieve superior performance with
adequate training data. Unlike traditional machine-learning methods, deep neural
networks typically use methods such as Word2Vec~\cite{mikolov2013distributed},
GloVe~\cite{peters2018dissecting}, or FastText~\cite{joulin2017bag} to represent
words as vectors. These methods create word embeddings by learning relationships
between words from large text corpora, eliminating the need for manual feature
engineering. Nevertheless, these methods represent all possible meanings of a
word in a single vector, making them unable to distinguish between different
word senses based on the surrounding context. Vaswani et al.
\cite{vaswani2017attention} introduced a new model able to provide
contextualized word representation called the Transformer. Originally designed
for neural machine translation, the Transformer addresses two limitations of
RNNs: lack of parallelization and handling of long-range dependencies. It relies
on the self-attention mechanism, which differentially weighs parts of the input.
Since it operates without recurrence, it is more parallelizable and
computationally efficient than RNNs.

In 2019, Devlin et al.~\cite{devlin2019bert} utilized parts of the original
Transformer architecture to develop BERT, achieving state-of-the-art results in
numerous NLP tasks. Performance of these large-scale language models heavily
depends on the underlying data used for pre-training. A homogeneous text corpus
generally leads to a poorer performing model compared to one trained on diverse
text corpora of high variance~\cite{martin2020camembert}. Initially, much of
BERT research was conducted with English texts, followed by efforts in
multilingual approaches~\cite{conneau2020unsupervised}. While multilingual
models were trained on extensive texts from numerous languages, it has been
shown that single language models outperform these and are even beneficial in
terms of efficiency, pre-training efforts, and downstream task performance as
they demand fewer computational resources and smaller datasets compared to the
extensive and diverse data required for multilingual
models~\cite{scheible2020gottbert, chan2020german, martin2020camembert}. In
particular, single-language models trained with the Open Super-large Crawled
ALMAnaCH coRpus (OSCAR)~\cite{suarez2019asynchronous} demonstrated strong
performance, benefiting from the corpus's size and variability. Notable examples
include CamemBERT~\cite{martin2020camembert} for French,
GottBERT~\cite{scheible2020gottbert} for German, and BERTje~\cite{de2019bertje}
for Dutch.

With the widespread adoption of Transformers in NLP tasks across various fields,
a trend towards domain-adapted language representation models has emerged. By
continuing the pre-training of models on in-domain data, researchers have been
able to improve the performance of models on specific tasks. In the biomedical
field, the pioneering and most recognized pre-trained model is
BioBERT~\cite{lee2020biobert}, which shares the same architecture as BERT.
Following a domain-adaptation strategy, BioBERT starts with BERT weights
pre-trained on general texts and then refines these weights using biomedical
corpora, surpassing the original model and achieving state-of-the-art
performance in numerous biomedical text mining tasks, such as clinical concept
recognition, gene-protein relation extraction, and biomedical question
answering. To gather sufficient open-source biomedical data, the authors
utilized repositories like PubMed~\cite{white2020pubmed} and PMC~\cite{pmcoa},
obtaining 4.5 billion words from abstracts and 13.5 billion words from full-text
articles. A similar method is employed by SciBERT~\cite{beltagy2019scibert},
which retains the original BERT configuration but substitutes the initial
general corpora with 1.14 million scientific articles randomly chosen from
Semantic Scholar. This dataset consists of 82\% broad biomedical domain papers
and 18\% computer science domain papers. By training from the ground up on
biomedical data, SciBERT can utilize a custom dictionary that better represents
the domain-specific word distribution. Med-BERT, introduced
by~\cite{liu2021med}, is the first model fully trained on hospital data,
particularly semi-structured electronic health records, leading to enhanced
performance in subsequent prediction models. These approaches have since been
refined, either by updating the model architecture to use BERT variants or by
expanding the biomedical corpus with additional sources beyond scientific
literature~\cite{huang2019clinicalbert, peng2019transfer}.

The extensive range of biomedical and clinical BERT-based models benefit from
the abundance of publicly available biomedical data in English, such as
MIMIC~\cite{johnson2023mimic,johnson2023mimicnote}, the largest open-access
dataset of medical records, and extensive repositories of biomedical scientific
literature~\cite{white2020pubmed}. However, most other languages lack access to
these valuable resources, making it challenging to achieve the same level of
performance as their English counterparts. Despite this, researchers from
various countries have endeavored to pre-train non-English biomedical models,
utilizing local and often non-public biomedical text collections. They have
either trained new models from scratch~\cite{akhtyamova2020named} or applied
biomedical domain adaptation to multilingual~\cite{rubel2020biobertpt} or
monolingual~\cite{copara2020contextualized} versions of BERT.

For what concerns the German language, advancements in medical language models
are significantly delayed and are often propelled solely by commercial software
or localized applications~\cite{starlinger2017improve}. Stringent data
protection laws impede data sharing, leading clinics to restrict data usage to
internal purposes~\cite{hellrich2015sharing}. These obstacles hinder the sharing
of datasets and models, as well as the organization of open challenges involving
German datasets. In spite of these challenges, there have been notable
initiatives in recent years: Datasets such as JSynCC~\cite{lohr2018sharing} and
GGPONC~\cite{borchert2022ggponc} have been released, containing German
biomedical language texts that are not subject to data protection concerns.
Recently, the introduction of the BRONCO150~\cite{kittner2021bronco150} corpus,
which includes de-identified discharge letters, and
GPTNERMED~\cite{frei2023gptnermed}, which leverages large language models, has
further expanded the availability of German medical text data. Additionally, the
CLEF eHealth challenge in 2019 provided a dataset of non-technical summaries of
animal studies to be classified according to the International Classification of
Diseases and Related Health Problems (ICD-10)~\cite{clef2019nts, clef2019test,
world1992icd}. A study by~\cite{sanger2019classifying} utilized the multilingual
BERT version (mBERT) to classify these summaries, demonstrating that mBERT
significantly outperformed a baseline Support Vector Machine model. To
incorporate advances in general German language models,
\cite{lentzen2022critical} introduced BioGottBERT, a model pre-trained on open
medical German texts from Wikipedia and scientific abstracts, which demonstrated
superior performance over its generalized counterpart GottBERT on medical tasks.
Subsequently, the authors of~\cite{bressem2024medbert} proposed medBERT.de, in
order to address the limited training data size and narrow scope on merely one
medical subarea by using 3.8 million radiology reports, achieving promising
results in classification tasks. While BioGottBERT was trained on a relatively
small corpus slightly less than 1 GB of text, medBERT.de significantly expanded
its training corpus to 10 GB, incorporating a wider variety of sources. However,
its BERT architecture has been surpassed by its optimized version RoBERTa as
recently demonstrated for German by the unpublished GeistBERT model. GeistBERT
reiterated on GottBERT~\cite{scheible2020gottbert}, by using Whole Word Masking
(WWM) and continued pre-training on a significantly more varied and larger
general-domain corpus, thereby establishing state-of-the-art performance on
various German NLP benchmarks.

\section*{Objectives}
In this work, we aimed to develop a new comprehensive German clinical language
model based on the RoBERTa architecture by building upon the foundation laid by
GeistBERT, hereinafter referred to as \textit{\ChristBERT}: \textbf{C}linical-
and \textbf{H}ealthcare-\textbf{R}elated \textbf{I}ssues and \textbf{S}ubjects
\textbf{T}uned BERT. The main emphasis of this work lies in the construction of
a large German pre-training corpus, encompassing a diverse range of biomedical
and clinical texts. These sources provided a broad spectrum of medical
language data, ensuring the model’s robustness and applicability. In order to
achieve this, we utilized a combination of mostly publicly available German
medical textual data and synthetic German domain texts by augmenting the corpus
with back-translated medical texts~\cite{edunov2018understanding}. This approach
involves translating a monolingual corpus using neural machine translation
models~\cite{ng2019facebook, costa2022no}, allowing us to leverage the vast
amount of public English medical texts available. Based on the constructed
corpus, we pre-trained \ChristBERT{} by using WWM and following three
different domain-adaptation strategies: (1) continued pre-training, (2)
pre-training from scratch, and (3) pre-training from scratch with additional
prior vocabulary adaptation. In order to investigate the effects of the
different domain-adaptation approaches, we evaluated the performance of the
resulting models on two domain-specific downstream tasks: named entity
recognition and classification. The downstream task performance has been
thoroughly evaluated and compared to existing medical and general-purpose German
language models.
