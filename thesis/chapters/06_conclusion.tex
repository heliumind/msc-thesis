\chapter{Conclusion} \label{chap:conclusion}

In this study, we have systematically investigated distinct domain adaptation
strategies for adapting language models to the German biomedical domain:
continued pre-training from a general-domain model, pre-training entirely from
scratch on biomedical data, and additionally adapting the vocabulary using
domain-specific BPE. Central to this effort was the curation of a large-scale
pre-training corpus, which prioritized data quantity and variety, while ensuring
the inclusion of mostly publicly available data. In order to mitigate the
scarcity of German biomedical data, we employed translation-based data
augmentation to increase the corpus size. Based on the resulting corpus, we
pre-trained three German clinical language models by following the
aforementioned adaptation strategies. These models were rigorously benchmarked
against existing general-purpose and medical German language models by
evaluating performance intrinsically via perplexity during pre-training and
extrinsically on a suite of domain-specific downstream tasks, namely NER and
text classification across five different German biomedical datasets.

The developed models set a new state-of-the-art for German clinical language in
our experimental setup. However, the empirical results demonstrated that no
single adaptation strategy universally outperformed the others. Continued
pre-training achieved low perplexity quickly and performed well on certain NER
tasks, benefiting from general language understanding. Pre-training from scratch
was more computationally intensive but excelled in specific classification
tasks, highlighting its strength in deep domain-specific feature learning.
Domain-adapted BPE vocabulary provided nuanced improvements, especially for
tasks requiring handling of specialized medical terminology, though its overall
impact varied across downstream tasks. We conclude that the optimal domain
adaptation strategy is context-dependent, involving a trade-off between
computational resources, training efficiency, and the specific requirements of
the target application.

Based on these findings and the limitations identified, several avenues for
future research emerge. Exploring architectures designed for longer sequence
lengths could enhance performance on tasks requiring document-level context,
common in clinical narratives. Expanding the pre-training corpus to include a
wider variety of medical subdomains and potentially incorporating mixed-domain
data could improve model generalizability. Finally, refining translation
methodologies for corpus augmentation and investigating the impact of translated
datasets on clinical applications remain important considerations for advancing
the field of German biomedical NLP. This work contributes state-of-the-art
German biomedical language models and provides valuable insights into domain
adaptation strategies, paving the way for future advancements in clinical text
processing and mining.
