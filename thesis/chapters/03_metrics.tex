\subsection{Evaluation Metrics}
Depending on the downstream task, different evaluation metrics are commonly used
to assess the performance of models. Since both NER and text classification are
framed as classification tasks, we focus on metrics for classification. In the
context of model evaluation, the terms \textbf{True Positive}, \textbf{False
Positive}, \textbf{True Negative} and \textbf{False Negative} are commonly
used~\cite{sokolova2009systematic}. These are defined as follows:

\begin{itemize}
    \item \textbf{True Positive} (TP): positive samples that are predicted as
    positive
    \item \textbf{False Positive} (FP): negative samples that are predicted as
    positive
    \item \textbf{True Negative} (TN): negative samples that are predicted as
    negative
    \item \textbf{False Negative} (FN): positive samples that are predicted as
    negative
\end{itemize}

The terminologies positive and negative simply refer to certain classes. A
fundamental measure for classification evaluation is the \textbf{confusion
matrix} which shows the relation between the predicted classes and the actual
classes. Thus, the correct predictions as well as the model errors can be
derived from this matrix. An example of such a confusion matrix in a
\textit{binary classification} scenario is shown in
Fig.~\ref{fig:confusion_matrix}.

\begin{figure}[htbp]
    \centering
    \def\myConfMat{{
    {1620,  100},
    {   13,1300},
    }}
\def\classNames{{"A","B"}}

\def\numClasses{2}

\def\myScale{1.3}
\begin{tikzpicture}[
    scale = \myScale,
    ]

\tikzset{vertical label/.style={rotate=90,anchor=east}}
\tikzset{diagonal label/.style={rotate=45,anchor=north east}}

\foreach \y in {1,...,\numClasses} %loop vertical starting on top
{
    % Add class name on the left
    \node [anchor=east] at (0.4,-\y) {\y};
    \foreach \x in {1,...,\numClasses}
    {
%---- Start of automatic calculation of totSamples for the column ------------   
    \def\totSamples{0}
    \foreach \ll in {1,...,\numClasses}
    {
        \pgfmathparse{\myConfMat[\ll-1][\x-1]}   %fetch next element
        \xdef\totSamples{\totSamples+\pgfmathresult} %accumulate it with previous sum
    }
    \pgfmathparse{\totSamples} \xdef\totSamples{\pgfmathresult}  % put the final sum in variable
%---- End of automatic calculation of totSamples ----------------
    
    \begin{scope}[shift={(\x,-\y)}]
        \def\mVal{\myConfMat[\y-1][\x-1]} % The value at index y,x (-1 because of zero indexing)
        \pgfmathtruncatemacro{\r}{\mVal}   %
        \pgfmathtruncatemacro{\p}{round(\r/\totSamples*100)}
        \coordinate (C) at (0,0);
        \ifthenelse{\p<50}{\def\txtcol{black}}{\def\txtcol{white}} %decide text color for contrast
        \node[
            draw,                 %draw lines
            text=\txtcol,         %text color (automatic for better contrast)
            align=center,         %align text inside cells (also for wrapping)
            fill=black!\p,        %intensity of fill (can change base color)
            minimum size=\myScale*10mm,    %cell size to fit the scale and integer dimensions (in cm)
            inner sep=0,          %remove all inner gaps to save space in small scales
            ] (C) {\r\\\p\%};     %text to put in cell (adapt at will)
        %Now if last vertical class add its label at the bottom
        \ifthenelse{\y=\numClasses}{
        \node [] at ($(C)-(0,0.75)$) % can use vertical or diagonal label as option
        {\x};}{}
    \end{scope}
    }
}
%Now add x and y labels on suitable coordinates
\coordinate (yaxis) at (-0.3,0.5-\numClasses/2);
\coordinate (xaxis) at (0.5+\numClasses/2, -\numClasses-1.25);
\node [vertical label] at (yaxis) {Predicted Class};
\node []               at (xaxis) {Actual Class};
\end{tikzpicture}

    \caption[Example of a confusion matrix for binary classification]{Example of
    a confusion matrix for binary classification Class value 1 corresponds to
    \textit{positive} and class value 2 to \textit{negative}.}
    \label{fig:confusion_matrix}
\end{figure}

The diagonal shows the correctly predicted classes, while the off-diagonal cells
the misclassified objects. The matrix shows the number of samples that fall into
the given categories, with true positives and true negatives on the diagonal and
false positives and false negatives on the off-diagonal. In the example of
Fig.~\ref{fig:confusion_matrix}, 99\% of the data points from class 1 and 93\%
of the samples from class 2 have been correctly classified. The respective
relative frequency is often visualized through coloring of the corresponding
cell. The darker the cell, the higher the number of data points predicted for an
actual class. The ideal confusion matrix of a model in this example would have
black cells on the diagonal and white cells in the remaining matrix. This would
mean that all predicted classes correspond to the actual classes. Through this
matrix representation, weaknesses of the model for certain classes can be
unveiled.

Such visually derived performance information from the confusion matrix can be
expressed in numerical metrics, which we will define in the following. The
ratio of the number of correct predictions over the total number of
predictions is referred to as \textbf{accuracy} and can be formulated as
follows:

\begin{equation}
    \text{Accuracy} = 
    \frac{\text{TP} + \text{TN}}{\text{TP} + \text{TN} + \text{FP} + \text{FN}}
\end{equation}

Evaluating a model solely on the accuracy metric can be misleading in cases of
imbalanced datasets. Therefore, two more metrics, namely \textbf{recall} and
\textbf{precision} can be incorporated to further analyze the performance of a
given model. These two metrics evaluate the performance on the classes
individually.

\textbf{Recall} evaluates how well the model can find all objects of a given
class. This can be mathematically formulated as:

\begin{equation}
    \text{Recall} = \frac{\text{TP}}{\text{TP} + \text{FN}}
\end{equation}

\textbf{Precision} on the other hand, measures how often a model is able to
predict a given class correctly:

\begin{equation}
    \text{Precision} = \frac{\text{TP}}{\text{TP} + \text{FP}}
\end{equation}

From the two latter metrics, the so-called \textbf{\ff{} score} can be computed.
The \ff{} score is defined as the harmonic mean of precision and recall:

\begin{equation} \label{eq:f1}
    \text{\ff} = 2 \cdot 
    \frac{\text{Precision} \cdot \text{Recall}}{\text{Precision} + \text{Recall}}
\end{equation}

We have seen classification metrics for binary classification so far. When
expanding recall, precision and \ff{} to \textit{multi-class} and
\textit{multi-label} classification scenarios (when more than two classes are
present), we have to account for multiple sets of TP, FP, and FN for each class.
As a result, we obtain different recall and precision values for each class. In
order to compile the individual values into a single value, we can average the
results. This is commonly done by averaging over the classes, i.e. \textbf{macro
averaging}, or over the individual samples, i.e. \textbf{micro averaging}
\cite{takahashi2022confidence}. Assuming $m$ classes and denoting the class
index with $i ,\ \forall i \ 1 \leq i \leq m$, then the averaged metrics are
defined as follows:

\paragraph{Macro averaging}
\begin{equation}
    \text{Recall}_\text{macro} = \frac{1}{m} \sum_{i=1}^{m} 
    \frac{\text{TP}_i}{\text{TP}_i + \text{FN}_i}
\end{equation}

\begin{equation}
    \text{Precision}_\text{macro} = \frac{1}{m} \sum_{i=1}^{m} 
    \frac{\text{TP}_i}{\text{TP}_i + \text{FP}_i}
\end{equation}

\paragraph{Micro averaging}
\begin{equation} \label{eq:micro_prec}
    \text{Recall}_\text{micro} = \frac{\sum_{i=1}^{m} \text{TP}_i}
    {\sum_{i=1}^{m} \text{TP}_i + \sum_{i=1}^{m} \text{FN}_i}
\end{equation}

\begin{equation} \label{eq:micro_rec}
    \text{Precision}_\text{micro} = \frac{\sum_{i=1}^{m} \text{TP}_i}
    {\sum_{i=1}^{m} \text{TP}_i + \sum_{i=1}^{m} \text{FP}_i}       
\end{equation}

TP\textsubscript{i} corresponds to the i-th diagonal cell in the confusion
matrix, FP\textsubscript{i} refers to the sum of the off-diagonal cell values of
the i-th row, FN\textsubscript{i} is the sum of the off-diagonal elements of the
i-th column~\cite{takahashi2022confidence}. The corresponding macro and micro
averaged \ff{} score is computed by applying Eq.~\ref{eq:f1} to the respective
recall and precision values. The averaging method should be chosen according to
the specific use case. Macro averaging takes the mean across all classes and
gives equal weight to each class. This is useful when working with imbalanced
datasets, as it penalizes errors in all classes equally, regardless of their
frequency. In contrast, micro averaging aggregates the contributions from all
classes and is typically used when the errors in infrequent classes are less
important compared to those in frequent classes~\cite{sokolova2009systematic}.
 

% \paragraph{NER Evaluation}
NER systems are typically evaluated using the \textbf{micro averaged \ff{}}
score, as it has established itself as the standard metric reported in the
literature and competitions, e.g. the CoNLL 2003 shared
task~\cite{tjong2003introduction}. In NER datasets, the number of entities often
follow a naturally imbalanced distribution, where some classes appear much more
frequently than specialized entity types. Micro averaging effectively weights
entity types according to their frequency in the dataset, which better reflects
real-world performance expectations~\cite{harbecke2022only}. Furthermore, NER
evaluation must consider both correct boundary detection and entity
classification. The micro averaged \ff{} score provides a unified measure that
captures both aspects by computing precision and recall based on the total
counts of true positives, false positives, and false negatives across all entity
types.

% \paragraph{Text Classification Evaluation}
For multi-label classification, evaluating prediction performance is inherently
different, as each prediction is associated with a set of labels rather than a
single label as in binary and multi-class classification. The quality of
multi-label classification is assessed through either partial or complete class
label matching~\cite{godbole2004discriminative}. The latter is often referred to
as \textit{exact matching} and extends binary and multi-label metrics to
multi-label classification by considering partially correct predictions as
incorrect. The \textit{hamming loss} forms the middle ground and measures the
fraction of wrongly predicted labels to the total number of labels. Both metrics
are dominated by the majority class and thus are not suitable for imbalanced
datasets. Given the significant class imbalance that the datasets used in this
thesis exhibit (refer to Sec.~\ref{sec:cls_datasets}), we will consider another
evaluation approach that incorporates fine-grained per-label statistics. This
approach essentially reduces the multi-label classification to $m$ individual
binary classification problems and evaluates the model's performance on each
label independently. Following the same rationale as for the NER evaluation, we
adopt the \textbf{micro-averaged \ff{}} score as introduced in
Eq.~\ref{eq:micro_prec}-\ref{eq:micro_rec}, for assessing text classification
performance. 

\subsection{Experimental Setup}

To assess performance, the fine-tuning experiments entailed a hyperparameter
optimization on each downstream task, tuning for batch size and learning rate.
Tab.~\ref{tab:hyperparams} shows the specific hyperparameters used to perform a
grid search, resulting in a total of 28 trials per task. The batch size and
learning rate sets are adapted from the evaluation setup of GeistBERT and were
extended with additional learning rate values. In each individual trial, all
benchmark datasets were fine-tuned with a warmup step ratio of 10\% for up to 30
epochs. During the grid search, the best model checkpoint was selected per epoch
based on the performance on the validation set of the downstream task. For NER
and classification tasks, we report the \textit{micro averaged} precision,
accuracy and \ff{} scores on each benchmark's test set. 

\begin{table}[htb]
    \centering
    \begin{tabular}{lc}
    \toprule
    \bfseries Parameter & \bfseries Values\\
    \midrule
    Learning Rate & \num{7e-5}, \num{5e-5}, \num{2e-5}, \num{1e-5}, \num{7e-6}, \num{5e-6}, \num{1e-6} \\
    Batch Size & 16, 32, 48, 64 \\
    \bottomrule
\end{tabular}

    \caption{Hyperparameters used in the grid search for the downstream tasks}
    \label{tab:hyperparams}
\end{table}

In contrast to perplexity as discussed in Sec.~\ref{sec:lm_eval}, downstream
task evaluation allow for a direct comparison of model efficacy across different
architectures and domains. Thus, we chose current state-of-the-art (SOTA) German
language models as baselines to benchmark the \ChristBERT{} models against,
including two domain-specific and two general-purpose Transformer models (refer
to Tab.~\ref{tab:models}). To facilitate performance comparison, all model and
downstream task combinations underwent the same hyperparameter tuning and
evaluation process as described above. The following provides a brief overview
of each model, with further design differences shown in
Tab.~\ref{tab:model_props} in the Appendix.

\begin{table}[htb]
    \centering
    \begin{tabular}{l rrr}
    \toprule
    \bfseries \multirow{2}{*}{Model} & 
    \bfseries \multirow{2}{*}{Type} & 
    \bfseries \multirow{2}{*}{Domain} &
    \bfseries Corpus \\
    & & & \bfseries Size (GB) \\
     \midrule
    \ChristBERT & RoBERTa & Biomedical & 1,482 + 13.5 \\
    \ChristBERT\textsubscript{scratch} & RoBERTa & Biomedical & 13.5 \\
    \ChristBERT\textsubscript{BPE} & RoBERTa & Biomedical & 13.5 \\
    medBERT.de~\cite{bressem2024medbert} & BERT & Biomedical & 10.3 \\
    BioGottBERT~\cite{lentzen2022critical} & RoBERTa & Biomedical & 145 + 0.8 \\
    GeistBERT & RoBERTa & General & 145 + 1,337 \\
    GeBERTa~\cite{dada2023impact} & DeBERTa & General & 167 \\
    \bottomrule
\end{tabular}

    \caption[Architecture, domain, and corpus size of evaluated models]{
    Architecture, domain, and corpus size of evaluated models. For \ChristBERT,
    BioGottBERT and GeistBERT, corpus size indicates the size of the initial +
    continuous pre-training corpus.}
    \label{tab:models}
\end{table}

\paragraph{medBERT.de} is based on the BERT~\cite{devlin2019bert} base
architecture and is specialized for the German medical domain. Similar to
\ChristBERT\textsubscript{BPE}, it was trained from scratch with a custom
domain-specific vocabulary on a large and diverse 10.3 GB corpus, comprising 4.7
million German medical documents from eleven different sources, including
articles from the German health web, scientific texts, medical books, and
real-world clinical data such as electronic health records and radiology reports
from Charité University Hospital. This substantial dataset translated into SOTA
performance on various medical benchmarks, particularly for longer and more
complex texts, such as NER and ICD-10 chapter classification from radiology
discharge summaries and surgical reports.

\paragraph{BioGottBERT} is a domain-adapted version of the unfiltered base
variant of the general German language model
GottBERT~\cite{scheible2020gottbert}, which itself is a RoBERTa-based model
trained on the German portion of the OSCAR~\cite{suarez2019asynchronous} corpus
(145 GB of general text). Similar to \ChristBERT, BioGottBERT was merely
\textit{continuously pre-trained} on biomedical German texts, specifically from
Wikipedia, scientific abstracts and drug leaflets. Despite being trained on a
relatively modest biomedical corpus of 809 MB, BioGottBERT demonstrated notable
improvements over its general-domain counterpart on a variety of medical NLP
tasks, including NER and classification problems. Our choice of evaluation
benchmarks closely aligns with the evaluation setup of BioGottBERT, which in
combination with an identical tokenizer to \ChristBERT{} allows for a direct
comparison of performance.

\paragraph{GeistBERT} is a general German language model utilizing the RoBERTa
base architecture. Following the \textit{continued pre-training} approach, it
was initialized with the best checkpoint of filtered GottBERT
\cite{scheible2020gottbert} (with 94,530 steps) and incrementally trained for
another 100,000 steps using Whole Word Masking (WWM) on a large-scale 1.3 TB
sized corpus consisting of partially deduplicated contemporary German
datasets~\cite{nguyen2023culturax, tiedemann2012parallel}. These include
massively crawled web data in addition to publicly accessible legal texts.
GeistBERT achieves SOTA performance, outperforming even larger models in German
NER, classification and natural language inference tasks. Further,
Longformer~\cite{beltagy2020longformer} and
Nyströmformer~\cite{xiong2021nystromformer} variants were derived, supporting
longer input sequences up to 8,000 tokens. As the general-domain counterpart of
the \ChristBERT{} models, GeistBERT was included to assess the impact of
domain-specific pre-training on downstream performance.

\paragraph{GeBERTa} is another general-domain model, but it utilizes the DeBERTa
\cite{he2021deberta} base architecture, which incorporates \textit{disentangled
attention} mechanisms for improved contextual understanding. Trained on 167 GB
of cross-domain German texts ranging from formal, informal, legal, medical and
literature data, the model performance was evaluated on general and medical NER,
sentiment and hate-speech detection as well as question-answering tasks. GeBERTa
offers a different architectural perspective compared to RoBERTa-based models
and is included to assess the impact of model architecture and cross-domain
pre-training data on downstream performance in both general and domain-specific
downstream tasks.

\subsubsection{Implementation Details}

The fine-tuning experiments were implemented utilizing the \textsc{Huggingface
Transformers}~\cite{wolf2020transformers} Python library and the \textsc{Neural
Network Intelligence}~\cite{nni} framework for hyperparameter tuning. The choice
of libraries was reinforced by their native support for the dataset
implementations~\cite{fries2022bigbio,lhoest2021datasets}. For classification,
model inputs were tokenized with truncation to, and padding up to the maximum
length of 512 tokens. In the case of NER, longer sequences were split into one
or more sequences of 64 tokens except for GGPONC, which was split into 128
tokens. These values correspond to the 95-th percentile of the sequence lengths
across each dataset's training, validation and test splits. The evaluation
metrics were computed using the \textsc{seqeval}~\cite{seqeval} Python library
for NER and the \textsc{sklearn}~\cite{pedregosa2011scikit} Python library for
classification. In \textsc{seqeval}, \textit{strict} evaluation mode was
applied, measuring both the correctness of the entity boundary and the entity
class. All experiments were conducted on consumer-grade hardware, specifically
an NVIDIA RTX 3090 GPU with 24 GB VRAM, ensuring reasonable training times for
practitioners. The total computation time for all experiments encompassing all
35 model-dataset combinations, amounted to approximately 6.74 days (refer to
Tab.~\ref{tab:eval_total_time} in the Appendix).
