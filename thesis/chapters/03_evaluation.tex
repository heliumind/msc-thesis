\section{Downstream Task Evaluation} \label{sec:fine-tuning}

With minimal adjustments to its architecture, pre-trained LMs can be utilized
for downstream applications by \textbf{fine-tuning} them on task-specific
datasets. In this kind of fine-tuning, application-specific circuitry (similar
to language modeling heads) is added on top of pre-trained models, taking their
output as its input. The fine-tuning process consists of continued training by
using the labels of the supervised datasets to \textit{fine-tune} the weights of
the pre-trained model and the additional circuitry.

To demonstrate the efficacy of our domain-adapted model in biomedical language
modeling, we fine-tune and evaluate the \ChristBERT{} models on two common
biomedical downstream tasks: Named entity recognition and text classification.

\subsection{Named Entity Recognition}

Named Entity Recognition (NER) is one of the most fundamental biomedical text
mining tasks as it is essential for extracting information from unstructured
clinical narratives such as drug names, symptoms, or diseases. Examples of NLP
applications building upon NER include question answering systems, machine
translation and anonymizing texts for privacy purposes.

A named entity generally refers to any element that can be identified by a
proper name, such as a person, location, or organization. Given a text or a
sequence of words $(w_1, \ldots, w_n)$, the task of NER is to identify and
classify these proper names within the text into predefined categories
$\mathcal{C} = \{c_1, \ldots, c_m\}$. The goal is to assign a label $l \in
\mathcal{C}$ to spans of words in the input sequence, indicating the specific
entity category to which it belongs. The NER task is typically framed as a
sequence labeling problem, where each token in the input sequence is assigned a
label from the fixed set of entity classes, while accurately identifying the
boundaries of named entities.

% The NER task is challenging due to the ambiguity in segmenting NER spans,
% determining which tokens constitute entities and which do not, given that most
% words in a text are not named entities. Another challenge arises from type
% ambiguity, where the same entity might belong to multiple categories,
% complicating the classification process.

\subsubsection{IOB2 Labeling Scheme}
In order to facilitate NER as a sequence labeling task, an appropriate
annotation scheme is required that captures both the boundaries and their
respective named entity types. One of the most widely used formats is the
\textbf{IOB2 labeling scheme}~\cite{ramshaw1999text}, which stands for Inside,
Outside, and Beginning. This scheme introduces an \texttt{O} (outside) label and
prefixes entity categories with \texttt{I} (inside) and \texttt{B} (begin). For
instance, in a NER task involving labels such as \texttt{PER} (person),
\texttt{ORG} (organization), \texttt{LOC} (location), the IOB2 scheme employs
the following labels: \texttt{O}, \texttt{I-PER}, \texttt{B-PER},
\texttt{I-ORG}, \texttt{B-ORG}, \texttt{I-LOC}, \texttt{B-LOC}, \texttt{I-MISC}
and \texttt{B-MISC}. Thus, the number of labels is $2m + 1$, where $m$ is the
number of entity types.

In practice, \texttt{B}-prefixed labels are used to tag the initial word of
entities, while \texttt{I}-prefixed labels are applied to all subsequent words
within the same entity. An example using the IOB2 format is shown in
Fig.~\ref{fig:iob2}. This labeling approach ensures clarity when annotating
adjacent entities, specifying which words belong to which entity. Without
prefixed labels, such differentiation would not be possible.

\begin{figure}[htbp]
    \centering
    \begin{tabular}{c|cccccccc}
    \bfseries Word & John & Smith & works & at & Microsoft & in & Redmond & . \\
    \bfseries IOB2 Label & \texttt{B-PER} & \texttt{I-PER} & \texttt{O} & \texttt{O} & \texttt{B-ORG} & \texttt{O} & \texttt{B-LOC} & \texttt{O} \\
\end{tabular}

    \caption{Example of IOB2 labeling scheme}
    \label{fig:iob2}
\end{figure}

\subsubsection{NER Datasets}
The following publicly available datasets were used as benchmarks for NER task
evaluation:

\paragraph{BRONCO150}
The \textit{BRONCO150}~\cite{kittner2021bronco150} corpus consists of randomly
shuffled sentences originating from 150 discharge summary letters. The clinical
notes are provided by the oncology departments of the Charité University
Hospital in Berlin and the Tübingen University Hospital. The text has been
anonymized to facilitate the sharing of the corpus. Accompanying the raw text
are three label classes: \textit{Medication}, \textit{Treatment} and
\textit{Diagnosis}.

\paragraph{GGPONC}
The \textit{GGPONC}~\cite{borchert2022ggponc} corpus is derived from the German
project focused on medical guidelines for oncology. It mainly consists of
free-form text excerpts of these guidelines, which are supplemented with manual
annotations created by the authors. With a total of 201,838 clinical named
entities from 10,193 recommendations and background texts, GGPONC constitutes
the largest freely distributed data set of semantically annotated German medical
texts. Given its source, the text style is anticipated to differ significantly
from typical clinical text. However, this origin allows for circumvention of
data-sharing constraints, such as the pseudonymization of patient-related
information.

There are two major versions of the corpus, from which we utilize the second
version in this thesis. The NER-related annotations include the label classes
\textit{Clinical Drug}, \textit{Diagnostic}, \textit{Nutrient / Body Substance},
\textit{Therapeutic}, \textit{External Substance}, \textit{Diagnosis /
Pathology}, and \textit{Other Finding} in the \textit{fine} mode, as well as the
label classes \textit{Finding}, \textit{Substance}, and \textit{Procedure} in
the \textit{coarse} mode. Additionally, there are two variations concerning the
length of individual entity spans. The \textit{short} annotation approach
results in shorter entity spans, while the \textit{long} annotation approach may
include additional semantically relevant information within the spans. For our
evaluation, we focus on the most demanding scenario, which involves eight
fine-grained entity classes and long annotation spans. This choice reflects the
experimental setup in \cite{bressem2024medbert}, allowing for better
comparability of model performance.

\paragraph{CARDIO:DE}
While the previous two corpora focused on the oncology domain, the
\textit{CARDIO:DE}~\cite{cardiode} corpus originates from 500 doctor's letters
in the cardiovascular domain at the Heidelberg University Hospital. The NER
annotation data include six label classes: \textit{ActiveIng}, \textit{Drug},
\textit{Duration}, \textit{Form}, \textit{Frequency}, and \textit{Strength}.
Additional annotation tasks, such as the segmentation of entire sections or
paragraphs, are also covered, but they are not relevant to the context of this
work. The authors released two subcorpora: \textit{CARDIO:DE} and
\textit{CARDIO:DE\textsubscript{EXP}}. \textit{CARDIO:DE\textsubscript{EXP}}
provides further label classes  \textit{Dosage}, \textit{Reason}, and
\textit{Route}, but they are marked as experimental due to significant
disagreements encountered during the annotation process. We omit these and
instead focus on performing NER on the main dataset with six label classes.

\begin{figure}[htbp]
    \centering
    \begin{subfigure}{0.49\textwidth}
        \centering
        % Prompt: Fill in the values in the tex table. The layout is the same.

\begin{tabular}{l ccc}
    \toprule
    \multirow{2}{*}[-0.5\dimexpr \aboverulesep + \belowrulesep + \cmidrulewidth]{\bfseries Entity} &
    \multicolumn{3}{c}{\bfseries Split} \\
    \cmidrule(lr){2-4}
    & Train & Val & Test \\
    \midrule
    Diagnosis & 3022 & 368 & 396 \\
    Medication & 1099 & 134 & 160 \\
    Treatment & 2036 & 234 & 263 \\
    \bottomrule
\end{tabular}

        \caption{BRONCO150}
    \end{subfigure}
    \hfill
    \begin{subfigure}{0.49\textwidth}
        \centering
        % Prompt: Fill in the values in the tex table. The layout is the same.

\begin{tabular}{l rrr}
    \toprule
    \multirow{2}{*}[-0.5\dimexpr \aboverulesep + \belowrulesep + \cmidrulewidth]{\bfseries Entity} &
    \multicolumn{3}{c}{\bfseries Split} \\
    \cmidrule(lr){2-4}
    & Train & Val & Test \\
    \midrule
    ActiveIng & 4803 & 699 & 598 \\
    Drug & 1372 & 154 & 154 \\
    Duration & 993 & 124 & 116 \\
    Form & 123 & 20 & 13 \\
    Frequency & 4097 & 596 & 457 \\
    Strength & 4004 & 575 & 447 \\
    \bottomrule
\end{tabular}

        \caption{CARDIO:DE}
    \end{subfigure} 
    \\[\baselineskip]
    \begin{subfigure}{0.8\textwidth}
        \centering
        % Prompt: Fill in the values in the tex table. The layout is the same.

\begin{tabular}{l rrr}
    \toprule
    \multirow{2}{*}[-0.5\dimexpr \aboverulesep + \belowrulesep + \cmidrulewidth]{\bfseries Entity} &
    \multicolumn{3}{c}{\bfseries Split} \\
    \cmidrule(lr){2-4}
    & Train & Val & Test \\
    \midrule
    Clinical Drug & 10973 & 2522 & 2219 \\
    Diagnosis / Pathology & 43249 & 9344 & 9997 \\
    Diagnostic & 17222 & 4035 & 3530 \\
    External Substance & 537 & 181 & 149 \\
    Nutrient / Body Substance & 2462 & 600 & 558 \\
    Other Finding & 31219 & 6520 & 7058 \\
    Therapeutic & 36288 & 7254 & 8421 \\
    \bottomrule
\end{tabular}

        \caption{GGPONC with \textit{fine}-grained entity classes and
        \textit{long} annotation spans}
    \end{subfigure}
    \\[\baselineskip] 
    \begin{subfigure}{0.8\textwidth}
        \centering
        % Prompt: Fill in the values in the tex table. The layout is the same.

\begin{tabular}{l rrr}
    \toprule
    \multirow{2}{*}[-0.5\dimexpr \aboverulesep + \belowrulesep + \cmidrulewidth]{\bfseries Class} &
    \multicolumn{3}{c}{\bfseries Split} \\
    \cmidrule(lr){2-4}
    & Train & Val & Test \\
    \midrule
    Trauma Surgery & 207 & 27 & 32 \\
    Ophthalmology & 31 & 3 & 2 \\
    Orthopedics & 217 & 32 & 33 \\
    Emergency Medicine & 87 & 9 & 11 \\
    Traumatology & 17 & 1 & 1 \\
    Anesthesiology & 30 & 3 & 2 \\
    \bottomrule
\end{tabular}

        \caption{JSynCC}
    \end{subfigure}
    \caption{Entity and class distributions of the downstream tasks}
    \label{fig:benchmark_stats}
\end{figure}

\subsection{Text Classification}

Text classification is the process of assigning one or more labels to an entire
sequence of text based on its content. If a document is categorized into one of
two possible labels, it is referred to as a \textbf{binary classification}
problem. In \textbf{multi-class classification}, there are more than two
classes, and each document is assigned to exactly one of these classes. In
contrast, \textbf{multi-label classification} allows a document to be assigned
to a set of multiple labels.

\subsubsection{Text Classification Datasets} \label{sec:cls_datasets}
In the following, we provide an overview of the text classification benchmarks
selected for evaluation as well as the preprocessing steps performed on the
datasets.

\paragraph{CLEF}
As part of the CLEF eHealth 2019~\cite{crestani2019experimental} challenge, the
organizers provided a dataset consisting of 8,385 German non-technical summaries
of animal experiments (NTS) sourced from the \textit{AnimalTestInfo}
\cite{animaltestinfo} database. In order to comply with EU regulations,
AnimalTestInfo tracks and publishes planned animal studies in Germany to provide
greater transparency and increase the protection of animal welfare. NTS is a
concise document that provides a description of the underlying assessment
process and its findings in a manner that is easily comprehensible to the
public. Each experiment includes a brief title, a description of expected
benefits, and details on the pressures and harm inflicted on the animals.
Additionally, the summaries outline strategies to minimize unnecessary harm and
improve animal welfare. Experts annotated each summary using the German version
of the ICD-10~\cite{graubner2013icd} classification system, with varying levels
of detail, e.g. chapter, group. Approximately two-thirds of the experiments are
labeled with a single disease, 10\% with multiple diseases, and the rest have no
disease annotations. For each disease, the complete hierarchical path in the
ICD-10 ontology is provided, including up to two parent groups and the chapter
of the annotated disease. About two-thirds of the summaries are annotated with
2-level paths, e.g. \texttt{I} and \texttt{B50-B64}, 20\% with 3- or 4-level
paths, e.g. \texttt{IV} and \texttt{E70-E90,E10-E14} or \texttt{II} and
\texttt{C00-C97,C00-C75,C15-C26}, and less than 1\% are annotated only with
chapters, e.g. \texttt{VI}. Given that each NTS has received zero or more ICD-10
codes as document-level label, we pose this corpus as a \textit{multi-label text
classification} task. The dataset is divided into a stratified training and
validation set of size 7,543 and 842, respectively~\cite{clef2019nts}. For the
final evaluation, the organizers used a hold-out test set consisting of 407
experiments~\cite{clef2019test}. To account for the highly imbalanced class
distribution, we filtered out documents whose labels occurred less than 25
times, similar to how it is done in~\cite{lentzen2022critical}. As a result, the
number of documents and the number of classes were reduced to 5,688 and 230,
respectively.

\paragraph{JSynCC}
The JSynCC corpus~\cite{lohr2018sharing} is the first publicly accessible
dataset featuring synthetically generated medical documents written in German
clinical language. It comprises 867 case reports made up by medical
professionals for educational purposes, which are sourced from 10 medical
textbooks. This approach allows for corpus publication without infringing on
data privacy laws. In order to adhere to the copyrighted source material,
instead of physical corpus distribution, JSynCC relies on sharing software to
reconstruct identical copies of the corpus based on electronic versions of the
source textbooks. Each case report is associated with one or more specialized
medical fields, which makes this dataset well-suited for \textit{multi-label
text classification} tasks. However, the class distribution is highly
imbalanced, with most labels being represented only a few times. To address the
scarcity of some classes, similar to the CLEF dataset, we only retained
documents whose labels occurred at least 25 times, reducing the number of
documents from 867 to 534 and the number of classes from 313 to 6:
\textit{Trauma Surgery}, \textit{Ophthalmology}, \textit{Orthopedics},
\textit{Emergency Medicine}, \textit{Traumatology} and \textit{Anesthesiology}.

\subsection{Dataset Preparation}

For all NER benchmarks, we utilized the \textsc{BigBIO}~\cite{fries2022bigbio}
library for biomedical datasets to benefit from harmonized dataset schemas,
standardized entity labels in the IOB2 format, and consistent tooling to access
and ingest data into machine learning workflows. Regarding classification
benchmarks, we used the \textsc{Huggingface Datasets} \cite{lhoest2021datasets}
library, which provides a unified interface for accessing and processing
datasets and forms the basis for \textsc{BigBIO}. Note that whenever available,
the pre-assigned training, validation and test splits of the datasets were
retained. For datasets without predefined splits, i.e. BRONCO150, CARDIO:DE and
JSynCC, stratified random partitioning was performed, allocating 80\%, 10\% and
10\% of the data for training, validation and testing, respectively.
Fig.~\ref{fig:benchmark_stats} depicts the entity or class distribution among
each dataset split for each benchmark. We exclude the CLEF eHealth 2019 dataset
from this figure, as it contains 230 possible classes.
