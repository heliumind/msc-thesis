\chapter{Discussion} \label{chap:discussion}

In this chapter, we summarize the main findings of our experiments and discuss
them in relation to the three investigated domain adaptation strategies in
Sec.~\ref{sec:findings}. We then contextualize our findings with other study
outcomes in the field of biomedical NLP in Sec.~\ref{sec:prior_work}. Finally,
Sec.~\ref{sec:limitations} reflects on the limitations of our study and suggests
directions for future research.

\section{General Findings} \label{sec:findings}

In this study, we systematically explored three complementary strategies for
domain adaptation of German biomedical language models: continued pre-training
from a general-domain model (\ChristBERT{}), pre-training from scratch
(\ChristBERT\textsubscript{scratch}), and vocabulary adaptation via
domain-specific subword tokenization (\ChristBERT\textsubscript{BPE}). All
models were pre-trained on a newly curated 13.5 GB biomedical corpus and
evaluated on downstream biomedical tasks, including named entity recognition
(NER) and text classification. Our experiments reveal several principal findings
regarding the three investigated domain adaptation strategies.

First, continued pre-training proved particularly effective in terms of
efficiency. \ChristBERT{} achieved the lowest perplexity and converged fastest,
underscoring the benefits of leveraging general-domain knowledge. This
advantage, however, did not always translate into downstream superiority for
initializing biomedical models. Instead, its performance varied among downstream
tasks: While \ChristBERT{} excelled in NER, particularly on GGPONC, it ranked
lowest on complex classification tasks such as CLEF, indicating that inherited
general-domain priors may not always be beneficial for complex classification
tasks.

Second, pre-training from scratch led to robust and often superior downstream
performance. \ChristBERT\textsubscript{scratch} achieved top results on text
classification tasks, particularly JSynCC, where it attained the highest \ff{}
score. This suggests that domain-exclusive representations learned from scratch
may offer advantages in classification scenarios requiring broader semantic
coverage and contextual generalization.

Third, domain-specific vocabulary adaptation (\ChristBERT\textsubscript{BPE})
yielded the strongest performance for entity-centric tasks. Despite higher
perplexity during pre-training, this variant excelled in NER tasks across all
datasets, achieving state-of-the-art results on BRONCO150 and CARDIO:DE.
However, its performance in classification tasks was less competitive,
indicating that the benefits of domain-optimized tokenization are most
pronounced in tasks sensitive to terminological precision and morphological
complexity.

Finally, comparisons to general-purpose language models highlighted the
importance of domain adaptation. While general models such as GeistBERT and
GeBERTa remained competitive on certain datasets like CARDIO:DE and CLEF, they
were consistently outperformed by the \ChristBERT{} variants on more specialized
or complex biomedical tasks. Furthermore, smaller-scale domain adaptation
efforts (e.g., BioGottBERT) could not match the performance gains achieved
through our larger corpus and comprehensive pre-training strategies.

In summary, our findings emphasize that no single adaptation strategy
universally outperforms the others. Continued pre-training offers rapid
convergence and strong generalization. From-scratch pre-training provides robust
performance for classification, while additional domain-specific vocabulary is
most beneficial for specialized tasks like NER. Our results highlight that the
suitability of domain-specific tokenization strategies, such as a custom BPE
vocabulary, is highly task-dependent. This suggests that domain-specific BPE
tokenization is especially beneficial for entity recognition, where accurate
boundary detection and handling of rare terms are critical. In contrast,
classification tasks often rely more on the model's ability to generalize over
broader semantic and syntactic patterns rather than fine-grained tokenization.
Thus, in such contexts, the rigid subword splits introduced by domain-specific
BPE may offer less benefit, or even introduce unnecessary complexity. These
observations emphasize the importance of aligning vocabulary adaptation
strategies not only with the domain but also with the linguistic properties and
demands of the target task.

\section{Findings in the Context of Prior Work} \label{sec:prior_work}

Our findings align well with and extend previous work on domain-adaptive
pre-training. The study~\cite{gururangan2020don} demonstrated that continued
pre-training yields significant gains for domain-specific tasks, especially when
the target domain is distant from the original pre-training corpus. Our results
confirm this for German biomedical NLP: continued pre-training (\ChristBERT) led
to rapid convergence and strong performance in complex NER tasks like GGPONC.
Furthermore, previous findings from~\cite{el2022re} suggest that training from
scratch can be competitive with, or even outperform, continued pre-training on
biomedical classification tasks. Our \ChristBERT\textsubscript{scratch} model
demonstrated this by excelling on both the JSynCC and CLEF classification
benchmarks. In their experiments, the authors of~\cite{el2022re} also observed
that medical-specific vocabularies lead to performance gains in downstream
domain tasks. Our results mirror these prior observations, with
\ChristBERT\textsubscript{BPE} achieving top results in NER, reinforcing the
idea that domain-aligned vocabulary improves handling of specialized
terminology.

Inspired by~\cite{edunov2018understanding}, we translated English medical texts
into German to address the scarcity of native-language biomedical corpora. This
strategy proved effective in terms of downstream task performance compared to
medBERT.de~\cite{bressem2024medbert}, which relied exclusively on original
German data. GeBERTa~\cite{dada2023impact}, which also leveraged translated
medical texts, achieved similarly strong results, particularly in classification
scenarios. Notably, even general-purpose models performed competitively on
classification tasks, underscoring that large-scale general-domain pre-training
enriched with some biomedical content remains a viable approach for such tasks.
Nevertheless, our findings support the approach of translation-based corpus
construction, especially for tasks like biomedical NER, where domain-specific
nuances and terminology require targeted representation learning and original
German resources remain limited.

While implementing the translation strategy for MIMIC-IV with LLaMA 3.1 and
Pubmed Central with NLLB 200, similar to~\cite{dada2023impact} we also observed
that the quality of the machine-translated data was sensitive to translation
settings. In particular with NLLB 200, we noticed that larger context sizes and
sequence lengths frequently resulted in degraded translation quality. Phenomena
such as stuttering and incoherent phrase repetition became evident, especially
in complex biomedical sentences. This degradation can stem from several factors
inherent to current translation models and LLMs used for translation. For
instance, generic LLMs, if configured incorrectly during inference (e.g.
insufficient context window sizes), may fail to attend to the entire input
sequence, effectively \textit{forgetting} earlier parts of the input and
producing incomplete or nonsensical translations. Likewise, many dedicated
translation models are trained on sentences. Consequently, their ability to
handle longer sequences degrades, as the positional embeddings beyond the
trained length are less reliable, leading to instability and errors in
translation~\cite{costa2022no}. To ensure the reliability of the translated
corpus, we therefore opted for a context size of 384 tokens for NLLB 200, which
offered a favorable balance between translation throughput and linguistic
accuracy, mitigating some of these input length-related issues.

\section{Limitations and Future Work} \label{sec:limitations}

While this study provides valuable insights into domain adaptation strategies
for German biomedical language models, several limitations remain and point to
promising directions for future research.

Our investigation was limited to the RoBERTa architecture, following the design
path of GeistBERT and GottBERT~\cite{scheible2020gottbert}. Although this
ensured comparability, alternative Transformer architectures, including the
recently introduced ModernBERT~\cite{warner2024smarter}, may offer performance
advantages in terms of computational efficiency. RoBERTa's maximum input size
limitation becomes particularly constraining in biomedical contexts, where
clinical documents such as patient records or scientific articles often involve
extended contexts. Future work should explore long-context Transformers.
Architectures such as Longformer~\cite{beltagy2020longformer} and
Nyströmformer~\cite{xiong2021nystromformer} could offer significant advantages
in tasks requiring document-level understanding or the resolution of complex
cross-sentence dependencies.

Furthermore, our findings indicate that training from scratch can be most
effective under certain conditions. This discrepancy highlights the need for
further investigation into the factors that influence the effectiveness of
training from scratch versus continued pre-training. Future work should focus on
exploring the specific scenarios and downstream tasks where training from
scratch might outperform continued pre-training. It would be valuable to conduct
a more detailed analysis of the trade-offs between computational resources,
training time, and performance gains. Understanding these dynamics could provide
insights into optimizing model training strategies for various applications.
Similarly, while our models build directly on GeistBERT and GottBERT in terms of
tokenizer design and vocabulary size, these decisions were not revisited for the
biomedical domain. Given the distinct lexical properties of medical language,
alternative vocabulary sizes or tokenization schemes might further optimize
model performance.

Moreover, the range of biomedical benchmarks used, while diverse, does not fully
reflect the variety of clinical language processing needs. Tasks involving
decision support, complex narratives, and clinical reasoning were
underrepresented, which will hopefully change in the near future with the
release of the German Medical Text Corpus
Project~\cite{meineke2023announcement}. Addressing these gaps is important,
especially in light of models like medBERT.de~\cite{bressem2024medbert}, which
explicitly targeted such scenarios. In addition, subdomains such as radiology,
psychiatry, and primary care were not systematically explored, limiting our
conclusions about generalizability.

Our corpus design also presents limitations. While our approach exclusively used
biomedical data, models like GeBERTa~\cite{dada2023impact} have demonstrated
that mixed-domain corpora can enhance generalization, particularly for tasks
that bridge specialized and general language. Investigating mixed corpus
strategies within the same RoBERTa architecture could therefore provide deeper
insights into optimal corpus design for domain-adaptive pre-training. Further
considerations result from performing translation to augment the pre-training
corpus. Although translation enabled the creation of a large biomedical corpus,
the quality of this synthetic data was not manually verified by healthcare
professionals and, as discussed, was sensitive to context size, with larger
sizes impairing coherence. Future work could investigate the effects of
translation quality and translation model behavior itself to assess and mitigate
such artifacts more systematically. Likewise, we did not systematically analyze
the individual contributions of the different data sources within our corpus; it
remains unclear to what extent the translated data specifically improved
performance compared to relying solely on the original German sources.
Additionally, de-identified datasets, i.e. MIMIC-IV, contain artifacts such as
anonymization masks, which are not typically found in natural prose, potentially
affecting performance on other types of text. As a byproduct of the translation
effort, we have obtained a large bilingual corpus of clinical texts (MIMIC-IV
Notes) and biomedical literature (PubMed Central). This corpus could be used to
train German-English translation models specialized in the biomedical domain,
which would be beneficial for direct clinical applications as well as future
corpus construction efforts. 

Lastly, one should be cautious when interpreting results in cases of data
leakage. Notably, GeBERTa’s pre-training corpus included CLEF data, possibly
giving it an unfair advantage in the corresponding classification task. This
data leakage limits direct comparability and highlights the need for careful
dataset selection in future evaluations.
