\chapter*{Project Plan}\label{chapter:schedule}

The preliminary schedule for the thesis is visualized in
Fig.~\ref{tab:schedule}. In total, five different work packages (WP) are shown.
These will be described briefly in the following.

\begin{table}[h]
    \centering
    \begin{tabular}{|>{\centering\arraybackslash}p{1.5cm}|>{\centering\arraybackslash}p{1.5cm}|>{\centering\arraybackslash}p{1.5cm}|>{\centering\arraybackslash}p{1.5cm}|>{\centering\arraybackslash}p{1.5cm}|>{\centering\arraybackslash}p{1.5cm}|>{\centering\arraybackslash}p{1.5cm}|}
    \hline
    \multicolumn{6}{|c|}{1st of October 2024 - 31st of March 2025} \\ \hline
    Oct. & Nov. & Dec. & Jan. & Feb. & Mar. \\ \hline
    \multicolumn{2}{|c|}{\cellcolor{lightgray} WP I}  & & & & \\ \hline
    & \multicolumn{2}{|c|}{\cellcolor{lightgray} WP II} & & & \\ \hline
    & &  \multicolumn{2}{|c|}{\cellcolor{lightgray} WP III} & & \\ \hline
    & & & & \multicolumn{1}{|c|}{\cellcolor{lightgray} WP IV} & \\ \hline
    & & \multicolumn{4}{|c|}{\cellcolor{lightgray} WP V} \\ \hline
    \end{tabular}
 
    \caption{Thesis schedule}
    \label{tab:schedule}
\end{table}

\subsection*{WP I - Creating Benchmark Pipeline}
After setting up the evaluation pipeline and finalizing the selection of medical
benchmarks, we will run them on existing models such as medBERT.de, GottBERT,
Bio-GottBERT, GeistBERT to establish a baseline

\subsection*{WP II - Creating Corpus}
This will be the most time-consuming part of the thesis. Particularly
challenging will be working with Springer Nature API and translating sources to
German. Pre-processing will be done to obtain their binary representations.

\subsection*{WP III - Training BERT Models}
4 different models will be trained with different vocabulary sizes and starting
points. The training will be done on the corpus created in the previous phase.
The training will reuse GeistBERT's optimizer parameters and be done on a GPU
cluster. 

\subsection*{WP IV - Evaluating BERT Models}
Pre-trained models will be fine-tuned on the benchmarks and evaluated on the
pipeline created in the first work package. The results will be compared to the
baselines and other state-of-the-art models obtained from WP I.

\subsection*{WP V - Writing Thesis}
This is the ongoing phase of writing the thesis. The chapters background and
related work  will be written in parallel to the first work package. The last
two months will be dedicated to writing the main part including methodology,
evaluation, results the conclusion.

