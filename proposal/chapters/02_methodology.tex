\chapter*{Methodology}\label{chapter:methodology}

ChristBERT will be built upon the foundation provided by the state-of-the-art
German language model GeistBERT. GeistBERT is an improved version of GottBERT
leveraging a more diverse and larger training data as well as recent
developments in pre-training methodologies such as Whole Word Masking (WWM). We
will utilize the Byte-Pair Encoding (BPE) vocabulary introduced in GottBERT as a
starting point, a tokenizer relying on subword units which are extracted by
performing statistical analysis of the training corpus.

Constructing ChristBERT will involve pre-training models from scratch
initialized with GeistBERT's checkpoints, ensuring that it is tailored to the
specific needs of the medical domain. An interesting aspect of this development
will be the creation of a domain-suitable vocabulary. The vocabulary size can be
seen as hyperparameter and therefore could be investigated. In our case, we
would refer to vocabulary sizes of 44k, 52k (used by GottBERT) and 60k. This
will ensure that the model is both comprehensive in its understanding of medical
language and efficient in processing it.

Consequently, the optimal plan is to pre-train and compare 4 different models in
total with the following configurations:

\begin{itemize}
    \item Based on GeistBERT using GottBERT vocabulary
    \item With an own specialized vocabulary in different sizes: 44k, 52k and
    60k (if resources suffice)
\end{itemize}

Pre-Training objective on all configurations will be masked language modelling
(MLM) using whole word masking (WWM) while utilizing
fairseq~\cite{ott2019fairseq} as tooling. Training parameters will be adapted
from GeistBERT's configuration.

\subsection*{Training Data}

Corpus creation will play a central role in the development of ChristBERT. The
corpus will be composed of a diverse array of sources, including articles from
information service provider for healthcare professionals Hpsmedia, German
medical articles from Wikipedia, PubMed abstracts and full texts, Springer
Nature articles and a crawled text corpora from the German Health Web inspired
by~\cite{zowalla2020crawling}. 

These sources will provide a broad spectrum of medical language data, ensuring
the model's robustness and applicability. To enhance the diversity and coverage
of the corpus, translation augmentation will be employed, allowing for the
inclusion of texts that may not originally be in German. The crucial and
challenging part of this thesis will be acquiring Springer Nature data with
their API.

In summary, the following sources could be used for corpus creation:

\begin{itemize}
    \item Medical Wikipedia
    \item Hpsmedia Publishing Group
    \item Springer Nature
    \item PubMed abstracts (augmented)
    \item PMC: PubMed full text articles (augmented)
    \item Internet corpus similar to sGHW~\cite{zowalla2020crawling}
    \item Mimic (augmented, if possible)
    \item PhD thesis of Charité (optionally)
    \item radiology reports of Charité (if accessible)
    \item Abstracts from LIVIVO search engine (if accessible)
\end{itemize}

\subsection*{Evaluation}

ChristBERT will be evaluated in a rigorous process, leveraging existing
codebases and evaluation pipelines where possible. The model's performance will
be evaluated on a variety of downstream tasks after being fine-tuned on them
respectively and compared with other state-of-the-art BERT models in the German
(medical) domain. We will be using experiments analogous to those conducted for
medBERT.de~\cite{bressem2024medbert}, as detailed on their Hugging Face page. A
grid search similar to the one from GottBERT will be conducted to optimize the
model's parameters focusing on batch size and learning rate. 

The evaluation will include assessments on datasets such as
BRONCO150~\cite{kittner2021bronco150}, GGPONC 2.0~\cite{borchert2022ggponc},
MANTRA GSC~\cite{kors2015mantragsc}, GERNERMED++~\cite{frei2022gernermed}, and
GPTNERMED~\cite{frei2023gptnermed}, ensuring that ChristBERT's performance is
thoroughly vetted across a range of relevant medical tasks. Which of these will
actually be used will be decided in the course of the project. The decisive
factors here are available resources and time as well as the characteristics of
the data sets. Additionally, the i2c2 benchmark for smoker status
classification, could be translated to German using NMT for another evaluation
dataset, specifically for medical text classification.
